\section{Theorie}
\label{sec:Theorie}
\subscetion{Wechselwirkung von Gammastrahlung mit Materie}
-Photo
-Compton
-Paarbildung
-Klein-Nishina-Formel
-Extinktionskoeffizient
\subsection{Gammaspektren}
-chrakteristische Linien
-moochromatisches Spektrum

\subsection{Statistik der Gammapeaks}
-Zählexperimente Poissonverteilt
-Absorptionswahrscheinlichkeit
-Effizienz/Vollenergienachweiswahrscheinlichkeit
\subsection{Halbleitergrundlagen}
-Energiebänder
-direkt/indirekt
-Dotierung
\subsection{Aufbau eines Germaniumdetektors}
-Aufbau/Zonen/Spannung
-Signalentstehung

\subsection{Signalverarbeitung}
-Preamp
-Shaping/Amp
-MCA
\subsection{Unschärfe der Photopeaks}
-Rauschen(Elektronik)
-thermische Übergänge im Detekor
-intrinsische Unschärfe
-unschärfe in Anzahl erzeugter und detektierter Elektronen
-Abschirmung

% \begin{equation}
%     \label{eq:}

% \end{equation}
% \begin{figure}[H]
%     \centering
%     \includegraphics[scale=0.5]{content/}
%     \caption{\cite{}.}
%     \label{fig:}
% \end{figure}

%\cite{}
