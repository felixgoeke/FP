\section{Theorie}
\label{sec:Theorie}
\subsection{Wechselwirkung von Gammastrahlung mit Materie}
Die Wechselwirkung von hochenergetischen Photonen aus der Gammastrahlung lässt sich auf 3 zentrale Prozesse reduzieren: den Photoeffekt, den Comptoneffekt und die Paarbildung.
In Abbildung \ref{fig:Wechselwirkung} sind die Prozesse schematisch dargestellt.
\begin{figure}[H]
    \centering
    \includegraphics[scale=1.0]{illustration/WWPhoton.png}
    \caption{a) Photoeffekt b) Comptoneffekt c) Paarbildung.\cite{Detektor}}
    \label{fig:Wechselwirkung}
\end{figure}
\noindent Beim Photoeffekt wird das Photon von einem Hüllenelektron vollständig absorbiert und gibt dabei seine gesamte Energie an das Elektron ab. Das Elektron wird dabei aus der Atomhülle gelöst und es entsteht ein freier Platz in der Hülle, 
wenn das Photon genug Energie besitzt, um das Elektron aus der Atomhülle zu lösen. Die Energie des Photons muss dabei mindestens so groß sein wie die Bindungsenergie des Elektrons.
Dies ist bei Photonen der Gammastrahlung der Fall. Dies ist ein Absorptionsprozess. Aufgrund der diskreten Energien der Photonen der Gammastrahlung, der dagegen vernachlässigbaren Bandlücke des Germaniumhalbeiters und 
der Tatsache, dass ein Elektron die volle Energie absorbiert, kann angenommen werden, dass die detektierte Energie dieses Elektron der gesamten 
Energie des Photons entspricht.

\noindent Der Comptoneffekt ist ein Streuprozess, bei dem das Photon an einem freien Elektron gestreut wird. Dabei gibt das Photon einen Teil seiner Energie an das Elektron ab und wird in einem Winkel $\theta$ gestreut.
Das Photon selbst existiert also weiter nach dem Stoß und das Elektron nimmt daher nur einen Teil der Photonenenergie auf. Gemäß der Formel 
\begin{equation}
    \label{eq:Compton}
    E' = \frac{E}{1+\frac{E}{m_0c^2}(1-\cos(\theta))}
\end{equation}
wird die Photonenenergie $E'$ minimal bei Rückstreuung. Daher definiert dieser Punkt die sogenannte Comptonkante, die die maximale Energie der detektierten Elektronen 
definiert, die beim Comptoneffekt beteiligt sind. Der differentielle Wirkungsquerschnitt gibt die Wahrscheinlichkeit an, für eine Streuung in einen bestimmten Raumwinkel.
Für den Comptoneffekt ist dieser Wirkungsquerschnitt gegeben durch die Klein-Nishina-Formel
\begin{equation}
    \label{eq:KleinNishina}
    \frac{d\sigma}{d\Omega} = \frac{r_e^2}{2[1+\epsilon(1-\cos(\theta))]^2}\left(1+cos^2(\theta)+\frac{\epsilon^2(1-\cos(\theta))^2}{1+\epsilon(1-\cos(\theta))}\right)
\end{equation}
mit $\epsilon=\frac{E}{m_0c^2}$ und dem klassischen Elektronenradius $r_e$. In Abbildung \ref{fig:Photo} ist der Wirkungsquerschnitt in Abhängigkeit vom Streuwinkel $\theta$ dargestellt.
In $\phi$-Richtung ist der Wirkungsquerschnitt konstant.
\begin{figure}[H]
    \centering
    \includegraphics[scale=0.3]{illustration/Klein-Nishina_distribution-en.svg.png}
    \caption{Wirkungsquerschnitt bei Streuung von Photonen an einem Elektron (Compton-Streuung).\cite {Klein}}
    \label{fig:Photo}
\end{figure}

\noindent Die Paarbildung ist ein Prozess, bei dem das Photon in der Nähe eines Atomkerns in ein Elektron-Positron-Paar umgewandelt wird. Dabei muss die Energie des Photons mindestens doppelt so groß sein wie die Ruhemasse des Elektrons.
zusätzlich kann dieser Prozess nur in der Nähe eines Atomkerns stattfinden, da der Impuls des Photons erhalten bleiben muss. Dieser Prozess wird erst bei Photonenenergien deutlich über $1\si{\mega\electronvolt}$ relevant.
Solch hohen Energien werden in diesem Versuch nicht detektiert.

\noindent In Abbildung \ref{fig:Extinktion} ist der Extinktionskoeffizient $\epsilon$ von Germanium in Abhängigkeit von der Energie der Gammastrahlung und der Art der Wechselwirkung dargestellt.
Er gibt an wie stark die Intensität der Gammastrahlung pro Wegstrecke mit $ I(z)=I_0\exp(-\epsilon z)$ abnimmt.
\begin{figure}[H]
    \centering
    \includegraphics[scale=1.0]{illustration/Extinktionskoeffizient.png}
    \caption{Extinktionskoeffizient von Germanium in Abhängigkeit von der Energie der Gammastrahlung und der Art der Wechselwirkung.\cite{GammaRay}}
    \label{fig:Extinktion}
\end{figure}
\noindent Bei Energien zwischen $100\si{\kilo\electronvolt}$ und $2\si{\mega\electronvolt}$ dominiert der Comptoneffekt, wobei auch der Photoeffekt noch als Vollenergienachweis genutzt werden kann.
Die Interaktionswahrscheinlichkeit steigt beim Comptoneffekt linear mit der Kernladungszahl $Z$ des Absorbermaterials an, während sie beim Photoeffekt quadratisch mit $Z$ ansteigt.
Germanium hat eine relativ hohe Kernladungszahl von $Z=32$, weshalb auch noch viele Photonen über den Photoeffekt detektiert werden können, welcher auf Grund der diskreten 
Energie der Elektronen das Spektrum der Probe am besten charakterisiert. 
\subsection{Gammaspektren}
Ein radioaktives Isotop kann beim Zerfall in andere Isotope in einen angeregten Zustand übergehen. Diesen kann der Atomkern durch spontane Emission eines 
hochenergetischen Photons verlassen. Da ein reines Isotop nur eine endliche Anzahl an Zerfallsmoden besitzt, ist das Spektrum dieser hochenergetischen 
Gammaphotonen charakteristisch für das jeweilige Isotop. In einem Detektor kann dann die Energie der Photonen gemessen werden und ein Spektrum aufgenommen werden.
In ein Abbildung \ref{fig:Spektrum} ist ein Beispiel eines solchen Spektrums dargestellt. Die charakteristischen Linien sind dabei die Vollenergiepeaks die 
entstehen, wenn die Photonen ihre Energie durch den Photoeffekt vollständig im Detektor deponieren. Die Ereignisse abseits der Peaks können 
auf Streuereignisse der Photonen an den Elektronen zurückgeführt werden.
\begin{figure}[H]
    \centering
    \includegraphics[scale=1.0]{illustration/LinienSpektrum.png}
    \caption{Beispiel eines Gammaspektrums mit charakteristischen Linien.\cite{GammaRay}}
    \label{fig:Spektrum}
\end{figure}
In Abbildung \ref{fig:Mono} ist ein monochromatisches Spektrum dargestellt. Bei Isotopen, die nur in einen angeregten Zustand übergehen können,
werden nur Photonen mit einer Energie detektiert. Dadurch lässt sich am gemessenen Spektrum gut erkennen über welche Prozesse die Strahlung Energie
an den Detektor abgegeben hat. Der Full-Energy-Peak ist dabei der Peak, der durch den Photoeffekt entsteht. Die Comptonkante beschreibt 
den maximalen Energieübertrag auf das Elektron bei Rückstreuung des Photons. Im Comptonkontinuum gibt es einen Backscatterpeak. Dieser entsteht 
durch Rückstreuung von Photonen am Gehäuse des Detektors, welche dann ihre Energie vollständig per Photoeffekt im Detektor deponieren.
Pile-Up beschreibt den Fall, dass zwei Photonen gleichzeitig detektiert werden und als ein Ereignis aufgenommen werden. Die gemeinsame Energie der beiden Photonen sorgt dann für weitere Ereignisse 
jenseits des Full-Energy-Peaks.
\begin{figure}{H}
    \centering
    \includegraphics[scale=1.0]{illustration/MonoSpektrum.png}
    \caption{Monochromatisches Gamma-Spektrum.\cite{GammaRay}}
    \label{fig:Mono}
\end{figure}
\noindent Detektiert wird die Anzahl von Ereignissen mit einer bestimmten Energie in einem festen Zeitintervall. Zählexperimente wie dieses
bei denen die Detektionswahrscheinlichkeit konstant bleibt während der Dauer sind immer Poissonverteilt.
Das heißt der Fehler der gezählten Ereignisse ist $\sqrt{N}$, wobei $N$ die Anzahl der Ereignisse ist. 
Bei bekanntem Extinctionskoeffizient kann über die Formel 
\begin{equation}
    \label{eq:Extinktion}
    W=P(d) = 1-\exp(-\epsilon d)
\end{equation}
die Wahrscheinlichkeit berechnet werden, dass ein Photon in einem Detektor mit Dicke $d$ detektiert wird.
Gleichzeitig kann experimentell ein Maß für die Vollenergienachweiswahrscheinlichkeit $Q$ bestimmt werden. Diese gibt an wie wahrscheinlich es ist, dass ein Photon, 
welches von einem bekannten Isotop in Richtung des Detektors emittiert wird und über den Photoeffekt ein Signal erzeugt, detektiert wird.
Es kann über die Formel 
\begin{equation}
    \label{eq:Q}
    Q = \frac{N_{\text{Peak}}}{A\cdot W\cdot t}\cdot\frac{4\pi}{\Omega}
\end{equation}
bestimmt werden. Dabei ist $N_{\text{Peak}}$ die Anzahl der Ereignisse im Peak, $A$ die Aktivität der Probe, $t$ die Messzeit, $\Omega$ der Raumwinkel und $W$ die Emissionswahrscheinlichkeit eines Photons mit dieser Energie.
Der Raumwinkel $\Omega$ ist in diesem experimentellen Aufbau durch den Radius des Detektors und den Abstand zur Probe gegeben.
\begin{figure}[H]
    \centering
    \includegraphics[scale=0.6]{illustration/Raumwinkel.png}
    \caption{Skizze des Versuchsaufbaus.}
    \label{fig:Raumwinkel}
\end{figure}
\noindent Der Raumwinkel ist gemäß den Größen aus Abbildung \ref{fig:Raumwinkel} und der Annahme das $\alpha$ klein ist gegeben durch
\begin{equation}
    \label{eq:Raumwinkel}
    \Omega = 2\pi\left(1-\frac{a}{\sqrt{a^2+r^2}}\right)
\end{equation}
mit $r=d/2$ als Radius des Detektors und $a$ als Abstand zur Probe. Der Abstand $a$ beträgt dabei $7.02\si{\centi\meter}$ von Probe zu einer Aluminiumabschirmung
und $1.5\si{\centi\meter}$ von der Abschirmung zum Detektor. Der Detektor hat einen Radius von $r=2.25\si{\centi\meter}$.
Damit ergibt sich ein Raumwinkel von $\Omega/4\pi=0.016$.
\subsection{Halbleitergrundlagen}
-Energiebänder
-direkt/indirekt
-Dotierung
\subsection{Aufbau eines Germaniumdetektors}
-Aufbau/Zonen/Spannung
-Signalentstehung

\subsection{Signalverarbeitung und Unsicherheiten}
-Preamp
-Shaping/Amp
-MCA
-Rauschen(Elektronik)
-thermische Übergänge im Detekor
-intrinsische Unschärfe
-unschärfe in Anzahl erzeugter und detektierter Elektronen
-Abschirmung

% \begin{equation}
%     \label{eq:}

% \end{equation}
% \begin{figure}[H]
%     \centering
%     \includegraphics[scale=0.5]{content/}
%     \caption{\cite{}.}
%     \label{fig:}
% \end{figure}

%\cite{}
