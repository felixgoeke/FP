\section{Diskussion}
  \label{sec:Diskussion}
Im folgenden werden die Ergebnisse der Messungen diskutiert.

Im Allgemeinen wurde bei jeder Messung der Peak Nahe des nullten Kanals nicht in die Auswertung einbezogen.
Dieser Peak tritt in allen Messungen, auch der Untergrundmessung auf, und scheint somit nicht Nuklid-spezifisch zu sein.
Der Peak ist vermutlich auf eine Ansammlung aller niedrigen Energien zurückzuführen, die nicht hoch genug für die Detektion sind.
Gleich verhält es sich mit den Signalen in den sehr hoben Kanälen, die ebenfalls nicht in die Auswertung einbezogen wurden.

\subsection{Energiekalibrierung}
Die Ergebnisse für die Kalibrierung zwischen Kanalnummer und Energie sind in Tabelle \ref{tab:kalibrierung} aufgeführt.
\begin{table}
  \centering
  \caption{Fitparameter der Kalibrierung zwischen Kanalnummer und Energie.}
  \label{tab:kalibrierung}
  \begin{tabular}{c c}
    \toprule
    $a$ & $b$ \\
    \midrule
    0.228334 \pm 0.0 & -148.659709 \pm 0.000593 \\
    \bottomrule
  \end{tabular}
\end{table}
Während der Proportionalitätsfaktor $a$ mit einer Unsicherheit von $0.0$ sehr genau bestimmt werden konnte, 
führt das Ergebnis für den Fitparameter $b$ zu unphysikalischen Ergebnissen, da Kanäle kleiner 676 negative Energien messen würden.
Ideal wäre, wenn der 0. Kanal eine Energie von 0 MeV messen würde.
Ursächlich könnte eine falsche Zuordnung der Literaturenergien zu den Peaks sein.
Da die Kalibrierung für alle weiteren Messungen verwendet wird, sind auch die Ergebnisse der weiteren Messungen mit Vorsicht zu genießen.

\subsection{Vollenergienachweiswahrscheinlichkeit}
Auffällig bei der Bestimmung der Vollenergienachweiswahrscheinlichkeit ist, 
dass die Effizienz $Q$ des Peaks der Kanalnummer 3340, welchem die Literaturenergie von $778.9045 \si{\kilo\electronvolt}$ zugeorndet wurde,
deutlich von der nichtlinearen Regression abweicht.
Dies könnte an einer falsch ermittelten Anzahl der Signale liegen.
Die Anzahl der Signale wurde mittels einer Gaußfunktion bestimmt, die Abbildung dieser ist im Anhang \ref{fig:Europium-Peak-3340} zu finden.
In der Abbildung ist zu erkennen, dass die gefittete Gaußfunktion allerdings gut zu den Messwerten passt.
Demnach ist eine weitere Ursache für die große Abweichung erneut eine falsche Zuordnung der Literaturenergie zu dem Peak, wie auch bei der Kalibrierung.

\subsection{Bestimmung der Halbwertsbreite und der Zehntelwertsbreite}
Der experimentell bestimmte Wert für das Verhältnis von Halbwertsbreite zu Zehntelwertsbreite beträgt $0.5476$ und 
liegt damit sehr nah an dem Proportionalitätsfaktor der gefitteten Gaußfunktion von $0.5467$.
Die relative Abweichung 
\begin{equation}
  \frac{\Delta x}{x} = \frac{0.5476 - 0.5467}{0.5467} = 0.16 \si{\percent}
\end{equation}
ist sehr gering und zeigt, dass die Gaußfunktion ein gutes Modell für die Messwerte ist.

\subsection{Untersuchung des Compton-Kontinuums}
Bei der Untersuchung des Comptonkontinuums ergab sich mittels der linearen Regression der Energiekalibrierung
für den Vollenergiepeak bei Kanalnummer 6420 eine Energie von $E_\text{Photo}=1317.2446\pm 0.0006 \si{\kilo\electronvolt}$.
Der Literaturwert beträgt $E_\text{Photo}=662 \si{\kilo\electronvolt}$.
Demnach ist die experimentell bestimmte Energie um den Faktor $1.99$ größer als der Literaturwert.
Diese fast exakte Abweichung um den Faktor 2 scheint mehr auf einen systematischen Fehler, vermutlich in der Auswertung selbst, 
als auf einen statistischen Fehler zurückzuführen zu sein.
Da der Fehler nicht gefunden werden konnte, 
wurde diese Auffälligkeit in den weiteren Auswertungen berücksichtigt, 
um dennoch sinnvolle Ergebnisse zu erhalten.
In der Abbildung \ref{fig:compton} ist zu erkennen,
dass die berechnete Kanalnummer für die Comptonkante höher ist als der Bereich, 
in dem die Anzahl der Signale des Comptonkontinuums deutlich abfällt.
Da die Comptonkante, anders als in der Theorie, nicht scharf ist, liegt das Ergebnis im Rahmen der Erwartungen.
Die berechnete Kanalnummer für den Rückstreupeak und die Kanalnummer, welche mittels der Peak-Funktion bestimmt wurde,
unterscheiden sich, ähnlich wie die Energie des Vollenergiepeaks, um den Faktor $1.96$.

\subsection{Bestimmung des Absorptionswahrscheinlichkeiten}
Weitergehend wurden die Absorptionswahrscheinlichkeiten für den Comptoneffekt und den Photoeffekt bestimmt
und mit den gemessenen Signalen verglichen.
Die Berechnung der Absorptionswahrscheinlichkeiten ergab, dass der Inhalt des Comptonkontinuums um den Faktor $24.9$ größer ist als der Inhalt des Vollenergiepeaks.
Jedoch ergaben die experimentell bestimmten Werte ein Verhältnis von $2.265$.
Dies weicht demnach stark von dem theoretischen Wert ab.
Eine Ursache für diese Abweichung kann darin liegen,
dass die $\gamma$-Quanten nach dem Compton-Effekt weiterhin Energie besitzen und somit auch noch in den Vollenergiepeak detektiert werden.

\subsection{Aktivitätsbestimmung}
Die Aktivitäten liegen, bis auf den Wert für den ersten Vollenergiepeak, in der gleichen Größenordnung.
Durch die starke Abweichung des ersten Vollenergiepeaks, ergibt sich eine große Unsicherheit der ermittelten Aktivität.
Die Ursache könnte darin liegen, dass der Peak fälschlicherweise der Probe zugeordnet wurde und eher dem Untergrund zuzuordnen ist.
Lässt man diesen Wert außer Acht, so ergibt sich eine Aktivität von $A = 112.21 \pm 4.75 \si{\becquerel}$ und besitzt eine deutlich geringere Unsicherheit.

\subsection{Nuklidbestimmung}
Wie bereits aufgeführt, scheint es einen Fehler bei der Energiekalibrierung zu geben.
Daher wurde, um dennoch die Energien der Peaks Nukliden zuordnen zu können, der Fehler der Kalibrierung berücksichtigt und der Faktor miteonbezogen.
Mit Berücksichtigung des genannten Faktors, ergeben sich die in Tabelle \ref{tab:unbekannt} aufgeführten Ergebnisse.
Die meisten Vollenergiepeaks wurden $^{229}Th$ zugeordnet. Ein Mutternuklid von Thorium ist $^{233}U$.
Dies im Zusammenhang mit der gelben Farbe der Probe, die auf Uran hinweist, könnte die Zuordnung der Peaks zu Thorium erklären.
Eine Energie wurde $^{233}Pa$ zugeordnet, welches ein Mutternuklid ebenfalls von $^{233}U$ ist.
Ein weiteres Nuklid, welches in der Probe enthalten sein könnte ist $^{232}U$.
Dieses zerfällt über $^{228}Th$ und $^{224}Ra$. Die Vollenergiepeaks dieser Nuklide können ebenfalls, jedoch mit größeren Abweichungen, den gemessenen Peaks zugeordnet werden.

\subsection{Zusammenfassung}
Insgesamt sind alle Ergebnisse, welche die Energiekalibrierung betreffen, mit Vorsicht zu genießen, da sich dort ein systematischer Fehler eingeschlichen hat.
Mit Betrachtung des dort genannten Faktors, sind die Ergebnisse jedoch sinnvoll und liefern plausible Ergebnisse.

