\section{Diskussion}
\label{sec:Diskussion}
Die Resultate der Messungen zeigen eine hohe Übereinstimmung mit den theoretischen Vorhersagen.
Der Schwellenstrom \ref{sec:threshold} ist durch eine deutliche Diskretion gekennzeichnet, 
welche den Laserbetrieb inittiert.
Das Absorptionsspektrum \ref{fig:absorption_gemessen} entspricht der theoretischen Erwartung \ref{fig:absorption_theoretisch}.
Alle vier Minima im Absorptionsspektrum sind deutlich zu erkennen.
Dies lässt den Schluss zu, dass die Modensprünge (Mode-Hopping) der Diode erfolgreich durch die Kalibrierung vermieden werden konnten.
Die Dreiecksspannung des Generators konnte durch den Ausgleich mit der zweiten Diode effektiv unterdrückt werden, 
sodass die Minima im Absorptionsspektrum bezüglich einer horizontalen Linie dargestellt werden konnten.
Die feine Abstimmung des Diodenlasers konnte durch die Befestigung der optischen Bauteile auf einem optischen Tisch erreicht werden.
Des weiteren wurde die Justierung der Bauteile, insbesondere der horizontalen Anpassung des Gitters, in kleinen Schritten durchgeführt, 
da jede ungenaue Justierung zu einer Störung in den Messungen führt und das Absorptionsspektrum gegebenfalls gar nicht darstellebar ist.

