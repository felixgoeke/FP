\section{Diskussion}
  \label{sec:Diskussion}
Abweichungen werden im Folgenden normiert auf den jeweiligen Theoriewert angegeben.
Zunächst konnte aus den Justageplots Informationen über die Probenbreite $D=\SI{22.44}{\milli\meter}$ und den Strahldurchmesser $d=\SI{0.16}{\milli\meter}$ gewonnen werden.
Die Probengröße weicht um $\SI{10.8}{\percent}$ von den Theoriewerten ab. Dies lässt sich aber damit erklären, dass der X-Scan keinen scharfen Abfall der Ränder aufweist und so die 
Postion der Ränder nicht exakt bestimmt werden kann. Aus den beiden Scans und der Rockingkurve sowie mit der theoretischen Probenbreite konnten 3 verschiedenen 
Werte für den Geometriewinkel bestimmt werden.
\begin{alignat}{3}
  \alpha_\text{g, Rocking} &= \SI{0.54(0)}{\degree} \quad & \alpha_\text{g, X,Z-Scan} &= \SI{0.41}{\degree} \quad & \alpha_\text{g, Theorie} &= \SI{0.46}{\degree} \\
\end{alignat}
Der Mittelwert ergibt sich dann zu $\alpha_\text{g} = \SI{0.47}{\degree}$ mit einer Abweichung des Mittelwerts von $\SI{0.04}{\degree}$.

Die aus dem Parratt-Algorithmus bestimmten Werte für die Rauigkeit und den Brechungsindex werden in Tabelle \ref{tab:Ergebnisse} mit den Theoriewerten verglichen.
Dabei konnten besonders für die Dispersion $\delta$ besonders gute Werte erzielt werden. Die Rauigkeit hat definitiv die richtige Größenordnung und die Schichtdicke weicht auch nicht allzu stark
von dem Wert, der aus den Abständen der Minima berechnet wurde, ab. Die Werte für die Absorptionskoeffizienten $\beta_1$ und $\beta_2$ weichen jedoch stark von den Theoriewerten ab.
Dies liegt zum Teil daran, dass $\beta_2$ sehr klein ist und daher kaum Einfluss auf die Reflektivität hat. Die Abweichung von $\beta_1$ könnte ein Produkt von mehreren ungenau eingestellten Parametern sein.
Der Fit passt zwar über die lange Distanz sehr gut, aber vor allem im Bereich des kritischen Winkels ist er noch nicht optimal. Eine deutlich zu große gemessene Absorption könnte 
auch darauf hindeuten, dass das Substrat einiges zusätzlich an Licht absorbiert, was zu einer effektiven Zunahme der einzelnen Absorptionskoeffizienten führt.
\begin{table}
    \centering
    \caption{Vergleich der Ergebnisse des Parrat Plots mit der Theorie.}
    \label{tab:Ergebnisse}
    \begin{tabular}{c c c c}
        \toprule
        Parameter & Parratt Algorithmus & Theoriewert \cite{m-tolan2013} & Abweichung\\
        \midrule
        $\delta_{\text{Si}}$ & \SI{7.1e-6}{} & \SI{7.6e-6}{} & \SI{6.6}{\percent} \\
        $\delta_{\text{Pol}}$ & \SI{3.5e-6}{} & \SI{3.5e-6}{} & \SI{0.0}{\percent} \\
        $\sigma_{\text{Si}}$ & \SI{11e-10}{\meter} & ~$\SI{10}{\angstrom}$ &  \\
        $\sigma_{\text{Pol}}$ & \SI{15e-10}{\meter} & ~$\SI{10}{\angstrom}$ &  \\
        $\beta_1$ & \SI{6e-7}{} & \SI{1.9e-7}{} & \SI{68.3}{\percent} \\
        $\beta_2$ & \SI{1e-8}{} & \SI{1.75e-8}{} & \SI{75}{\percent} \\
        $d$ & \SI{8.1e-8}{\meter} & \SI{8.9e-8}{\meter} & \SI{9.0}{\percent} \\
        \bottomrule
    \end{tabular}
\end{table}
